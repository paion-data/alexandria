\documentclass{./antiquebook/antiquebook}

\usepackage{fontspec}
\setmainfont[ExternalLocation=fonts/]{porson.ttf}
\newfontfamily{\Ubuntu}{Ubuntu}

% this clashes with antiquebook
% \usepackage[left=1in, right=1in, top=1in, bottom=1in, headsep=.5in]{geometry}

\usepackage{fancyhdr}
\pagestyle{fancy}
\lhead{\href{https://github.com/QubitPi/aristotle}{\includegraphics[scale=0.05]{github.png}}}
\chead{
    \definecolor{mygreen}{HTML}{00AA00}
    \raisebox{.5\fontcharht\font`0}{\pgfornament[width=1.5cm, color=mygreen]{11}}
    {\Ubuntu Aristotle - Logic I: Categories, On Interpretation, Prior Analytics}
    \raisebox{.5\fontcharht\font`0}{\pgfornament[width=1.5cm, color=mygreen]{14}}
}
\rhead{\href{https://qubitpi.org/}{\includegraphics[scale=0.05]{logo-8th-version.png}}}
\lfoot{}
\cfoot{}
\rfoot{\thepage}
\renewcommand{\headrulewidth}{0.4pt}
\renewcommand{\footrulewidth}{0.4pt}

\setlength{\parindent}{0pt}

\usepackage{hyperref}
\hypersetup{
    colorlinks=true,
    linkcolor=blue,
    anchorcolor=blue,
    urlcolor=blue
}

\usepackage{graphicx}
\usepackage{float}
\graphicspath{ {./img/} }

\usepackage{tikz}
\usetikzlibrary{
    calc,
    hobby,
    quotes,
    angles,
    matrix,
    decorations,
    arrows.meta,
    decorations.markings,
    decorations.pathmorphing,
    decorations.pathreplacing
}
\usepackage[most]{tcolorbox}
\tcbuselibrary{skins}
\tcbuselibrary{raster}
\newtcbox{\roundinlinebox}[1][red]{
    on line,
    arc=7pt,
    colback=#1!10!white,
    colframe=#1!50!white,
    before upper={\rule[-3pt]{0pt}{10pt}},
    boxrule=1pt,
    boxsep=0pt,left=6pt,right=6pt,top=2pt,bottom=2pt
}

\usepackage{pgfornament}


\begin{document}
    Ὁμώνυμα λέγεται ὧν ὄνομα μόνον κοινόν, ό δὲ κατὰ τοὔνμα λὀγος τῆς οὐσίας, οἷον ζῷον ὅ τε ἄνθρωπος καὶ τὸ
    γεγραμμένον. τούτων γὰρ ὄνομα μόνον κοινόν, ὁ δὲ κατὰ τοὔνομα λόγος τῆς οὐσίας ἕτερος· ἃν γάρ τις ἀποδιδῷ
    τί ἰστιν αὐτῶν ἑκατέρῳ τὸ ζῴῳ εἶναι, ἴδιον ἑκᾰτέρου λόγον ἀποδώσει. συνώιυμα δὲ λέγεται ὧν τό τε ὅνομα κοινὸν καὶ ὁ
    κατὰ τοὔνομα λόγος τῆς οὐσίας ὁ αὐτὸς, οἷον ζῷον ὅ τε ἄνθρωπος καὶ ὁ βοῦς. ὁ γὰρ ἄνθρωπος καὶ ὁ λόγος δὲ τῆς οὐσίας
    ὁ αὐτός· ἐὰν γὰρ ἀποδιδῷ τις τὸν ἑκατέρου λόγον, τί ἐστιν αὐτῶν ἑκατέρῳ τὸ ζῴῳ εἶναι, τὸν αὐτὸν λόγον
    ἀποδώσει. παρώνυμα δὲ λέγειται ὅσα ἀπὸ τινος διαφέροντα τῇ πτώσει τὴν κατὰ τοὔνομα προσηγορίαν ἔχει, οἷον ἀπὸ τῆς
    γραμματικῆς ὁ γραμματικὸς καὶ ἀπὸ τῆς ἀνδρείας ὁ ἀνδρεῖος.
\end{document}
